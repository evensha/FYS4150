\documentclass[12pt, a4paper]{article}
\usepackage[print,sort]{standalone}
\usepackage[T1]{fontenc}
\usepackage[utf8]{inputenc}
\usepackage[english]{babel}
\usepackage{graphicx,float}
\usepackage{amssymb}
\usepackage{amsmath,cancel}
\usepackage{mathrsfs}
\usepackage{epstopdf}
\usepackage{subcaption}
\usepackage{slashed}
\usepackage{hhline}
\usepackage[margin=1.2in]{geometry}
\usepackage[hidelinks]{hyperref}
\usepackage{wrapfig}


\hfuzz=5pt


\begin{document}

\begin{titlepage}
\begin{center}
\vspace*{3cm}
\Huge
\textbf{Project 4} \\
\Large  
FYS4150 Computational Physics 
\vspace*{3cm} \\ 

Even S. Håland 
\vspace*{5cm} \\

\normalsize
\section*{Abstract}

\end{center}
\end{titlepage}

\section{Introduction}

The aim of this project is to do Monte Carlo simulations of magnetic systems described by the 
two-dimensional Ising model. The Ising model describes binary systems, that is a collection of objects 
with only two possible states each. Here we will take these two states to be spin-up and spin-down. In
particular we will study phase transitions in such systems. 

We will start by studying some analytical solutions to the Ising model by considering a $2\times 2$ 
lattice. Then we will make a code that simulates the evolution of such systems for larger lattices. We 
will see how the system evolves depending on temperature and the initial state of the system, and we 
will finally study phase transitions, and extract the critical temperature.  

In our simulations we will use the Metropolis algorithm for accepting or rejecting the Monte Carlo 
trials. We will also, as these simulations are quite time consuming, parallelize the code, so that 
we can distribute our simulation to multiple processors. 

\section{Theoretical framework}

\subsection{The Ising model}

As mentioned in the introduction the Ising model describes a collection of objects, where each object 
can take two values, i.e. spin-up and spin-down, parametrized by $s=\pm 1$. We will consider a 
two-dimensional model with the spins organized in a square lattice with dimension $L$, so the total 
number of spins is given by $N = L\times L$. The two key quantities for describing our system 
will be energy and magnetization. The total energy of this system (with no external magnetic field) 
is given by 
\begin{align}
E = -J\sum_{<kl>}^{N} s_k s_l, 
\end{align}
where $J$ is a coupling constant and $<kl>$ indicates that we sum over neighbouring spins only, i.e. 
one object is only "talking" to its closest neighbours. The total magnetization of the system is simply 
given by the sum over all spins: 
\begin{align} 
M = \sum_k^N s_k. 
\end{align}

A natural next step is to calculate expectation values within this theory, i.e. mean energy, 
$\left\langle E \right\rangle$, and mean magnetization $\left\langle M \right\rangle$. To be able to do 
so we need to introduce some concepts from statistical physics.  

\subsection{Elements from statistical physics}

In order to calculate expectation values we need a probability distribution. In our case this is the 
Boltzmann distribution, given as 
\begin{align*}
P_i(\beta) = \frac{e^{-\beta E_i}}{Z}, 
\end{align*}
where $\beta = 1/k_B T$ (where $k_B$ is Boltzmann's constant and $T$ is temperature), $E_i$ is the 
energy of state $i$ and $Z$ is the partition function. The partition function is given as 
\begin{align*}
Z = \sum_{i=1}^n e^{-\beta E_i}, 
\end{align*}
where the sum runs over all possible (micro)states of the system. The expectation value of some 
general quantity $A$ can now be calculated as 
\begin{align*}
\left\langle A \right\rangle = \sum_{i=1}^n A_i P_i(\beta) = \frac{1}{Z} \sum_{i=1}^n A_i e^{-\beta E_i}, 
\end{align*} 
and the variance of $A$ is given as 
\begin{align*}
\sigma_A^2 = \left\langle A^2 \right\rangle - \left\langle A \right\rangle^2. 
\end{align*}
By considering the variances of $E$ and $M$ we can also calculate specific heat, $C_V$, and 
susceptibility, $\chi$, as 
\begin{align*}
C_V = \frac{1}{k_BT^2}\sigma_E^2 \quad \text{and} \quad \chi = \frac{1}{k_B T}\sigma_M^2. 
\end{align*}

\subsection{Analytical solutions}

Using the above described framework we can study the Ising model for a $2\times 2$ lattice analytically. 
This means that we have $4$ objects which each can have $s=\pm 1$. Both in this and later analysis we 
will make use of so-called periodic boundary conditions, which means that we let the first and last 
spins in a lattice row be neighbours when summing up the energy. 

We label the spins in the $2\times 2$ lattice with $k \in \left\lbrace 1,2,3,4\right\rbrace$, and the 
energy is then given by 
\begin{align*}
E & = -J(s_1 s_2 + s_2 s_1 + s_1 s_3 + s_3 s_1 +s_2 s_4 +s_4 s_2 + s_3 s_4 + s_4 s_3) \\
  & = -2J(s_1 s_2 + s_1 s_3 + s_2 s_4 + s_3 s_4),  	
\end{align*}  
and the magnetization as 
\begin{align*}
M = s_1 + s_2 + s_3 + s_4. 
\end{align*}

\begin{table}[ht!]
\caption{Degeneracy, energy and magnetization for the different spin configurations in the $2\times 2$
lattice.}
\label{tab:2times2}
\begin{center}
\begin{tabular}{cccc} \hline\hline
$\#$ spin-up & Degeneracy & $E$ & $M$ \\ \hline
$4$ & $1$ & $-8J$ & $4$ \\ 
$3$ & $4$ & $0$ & $2$ \\
$2$ & $2$ & $8J$ & $0$ \\  
$2$ & $4$ & $0$ & $0$ \\ 
$1$ & $4$ & $0$ & $-2$ \\ 
$0$ & $1$ & $-8J$ & $-4$ \\ \hline
\end{tabular}
\end{center}
\end{table}

Table \ref{tab:2times2} lists the possible values for these quantities for all possible spin 
configurations, along with the degeneracy for each spin configuration. Using this we find that the 
partition function is given as 
\begin{align*}
Z = 12 + 2e^{8J\beta} + 2e^{-8J\beta} = 12 + 4 \cosh(8J\beta). 
\end{align*} 


\section{Code}

All code written for the project can be found in the following git-repository:  \vspace{0.5cm} \\ 
\fbox{
\href{https://github.com/evensha/FYS4150/tree/master/Project4/Programs}
{https://github.com/evensha/FYS4150/tree/master/Project4/Programs} 
} \vspace{0.5cm} \\ 


\section{Summary and conclusions}

\begin{thebibliography}{40}

\bibitem{Lecture Notes} M. Hjort-Jensen (2015), \textit{Computational Physics - Lecture Notes Fall 2015}, 
Department of Physics, University of Oslo. \\ 
\href{https://github.com/CompPhysics/ComputationalPhysics/blob/master/doc/Lectures/lectures2015.pdf}
{https://github.com/CompPhysics/ComputationalPhysics/blob/master\\/doc/Lectures/lectures2015.pdf}

\bibitem{Armadillo} C. Sanderson, R. Curtin (2016), Armadillo: a template-based C++ library for linear 
algebra, \textit{Journal of Open Source Software}, Vol. 1, p. 26.  

\end{thebibliography} 

\end{document}
