\documentclass[12pt, a4paper]{article}
\usepackage[print,sort]{standalone}
\usepackage[T1]{fontenc}
\usepackage[utf8]{inputenc}
\usepackage[english]{babel}
\usepackage{graphicx,float}
\usepackage{amssymb}
\usepackage{amsmath,cancel}
\usepackage{mathrsfs}
\usepackage{epstopdf}
\usepackage{subcaption}
\usepackage{slashed}
\usepackage{hhline}
\usepackage[margin=1.2in]{geometry}
\usepackage[hidelinks]{hyperref}
\usepackage{wrapfig}


\hfuzz=5pt


\begin{document}

\begin{titlepage}
\begin{center}
\vspace*{3cm}
\Huge
\textbf{Project 4} \\
\Large  
FYS4150 Computational Physics 
\vspace*{3cm} \\ 

Even S. Håland 
\vspace*{5cm} \\

\normalsize
\section*{Abstract}

\end{center}
\end{titlepage}

\section{Introduction}

The aim of this project is to do Monte Carlo simulations of magnetic systems described by the 
two-dimensional Ising model. The Ising model describes binary systems, that is a collection of objects 
with only two possible states each. Here we will take these two states to be spin-up and spin-down. In
particular we will study phase transitions in such systems. 

We will start by studying some analytical solutions to the Ising model by considering a $2\times 2$ 
lattice. Then we will make a code that simulates the evolution of such systems for larger lattices. We 
will see how the system evolves depending on temperature and the initial state of the system, and we 
will finally study phase transitions, and extract the critical temperature.  

In our simulations we will use the Metropolis algorithm for accepting or rejecting the Monte Carlo 
trials. We will also, as these simulations are quite time consuming, parallelize the code, so that 
we can distribute our simulation to multiple processors. 

\section{Theoretical framework}

\subsection{The Ising model}

As mentioned in the introduction the Ising model describes a collection of objects, where each object 
can take two values, i.e. spin-up and spin-down, parametrized by $s=\pm 1$. We will consider a 
two-dimensional model with the spins organized in a square lattice with dimension $L$, so the total 
number of spins is given by $N = L\times L$. The two key quantities for describing our system 
will be energy and magnetization. The total energy of this system (with no external magnetic field) 
is given by 
\begin{align}
E = -J\sum_{<kl>}^{N} s_k s_l, 
\end{align}
where $J$ is a coupling constant and $<kl>$ indicates that we sum over neighbouring spins only, i.e. 
one object is only "talking" to its closest neighbours. The total magnetization of the system is simply 
given by the sum over all spins: 
\begin{align} 
M = \sum_k^N s_k. 
\end{align}
In order to further evaluate this model we need to introduce some concepts from statistical physics. 

\subsection{Elements from statistical physics}



\subsection{Analytical solutions}

\section{Code}

\section{Results}

\section{Summary and conclusions}

\end{document}
