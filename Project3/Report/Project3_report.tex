\documentclass[12pt, a4paper]{article}
\usepackage[print,sort]{standalone}
\usepackage[T1]{fontenc}
\usepackage[utf8]{inputenc}
\usepackage[english]{babel}
\usepackage{graphicx,float}
\usepackage{amssymb}
\usepackage{amsmath,cancel}
\usepackage{mathrsfs}
\usepackage{epstopdf}
\usepackage{subcaption}
\usepackage{slashed}
\usepackage{hhline}
\usepackage[margin=1.2in]{geometry}
\usepackage[hidelinks]{hyperref}
\usepackage{wrapfig}


\hfuzz=5pt


\begin{document}

\begin{titlepage}
\begin{center}
\vspace*{3cm}
\Huge
\textbf{Project 3} \\
\Large  
FYS4150 Computational Physics 
\vspace*{3cm} \\ 

Even S. Håland 
\vspace*{5cm} \\

\normalsize
\section*{Abstract}

The motion of the planets in the solar system is governed by gravitational forces between the planets 
and the sun, and between the planets themselves. This can be expressed mathematically as a set of coupled 
differential equation. In this project we solve these equations, mainly by using the velocity Verlet 
algorithm. When doing so we use an object oriented approach, and two classes are developed; a 
\textit{planet} class and a \textit{solver} class. We find that the problem can be solved in a quite nice 
and compact way, and we get quite good estimates of the planetary orbits.  

\end{center}
\end{titlepage}

\section{Introduction}



\section{Modelling planetary motion}

Newtons law of gravitation states that the gravitational force between two objects, with mass $M$ and $m$
respectively, is given by 
\begin{align}
F_G = \frac{GMm}{r^2}, 
\end{align}
where $G$ is the gravitational constant and $r$ is the distance between the two objects. 

We start by considering a system with only two objects, namely the Sun (with mass $M_{\odot}$) and the 
Earth (with mass $M_{\text{ Earth}}$), and we assume that the Sun is fixed, so the only motion we have to
care about is that of the Earth. We also assume that the motion of the Earth is co-planar, and take this 
to be the $xy$-plane, with the Sun in the origin. When writing the programs and implementing the 
algorithm we actually work in $3D$, but extending from two to three dimensions is quite trivial.  

The forces acting on the Earth in the $x$- and $y$-direction are then given by 
\begin{align*}
F_{G,x} = - F_G \cos\theta  = - \frac{GM_{\odot} M_{\text{Earth}}}{r^2}\cos\theta 
							= - \frac{GM_{\odot} M_{\text{Earth}}}{r^3}x 
\end{align*} 
and 
\begin{align*}
F_{G,y} = - F_G \sin\theta  = - \frac{GM_{\odot} M_{\text{Earth}}}{r^2}\sin\theta 
							= - \frac{GM_{\odot} M_{\text{Earth}}}{r^3}y,  
\end{align*}
where we have used the relations $x = r\cos\theta$ and $y = r\sin\theta$. From Newtons second law we know 
that the accelerations, $a_x$ and $a_y$, are given as 
\begin{align*}
a_x = \frac{d^2x}{dt^2} = \frac{F_{G,x}}{M_{\text{Earth}}} \quad \text{and} \quad 
a_y = \frac{d^2y}{dt^2} = \frac{F_{G,y}}{M_{\text{Earth}}}. 
\end{align*} 
We also know that the acceleration is the time derivative of the velocity, $v$, which again is 
the time derivative of the position, meaning that we can express the equations of motion as two 
coupled first order differential equations in each dimension:  
\begin{align*}
v_x = \frac{dx}{dt}, \quad v_y = \frac{dy}{dt}, \quad 
a_x = \frac{dv_x}{dt} = -\frac{GM_{\odot}x}{r^3}, \quad a_y = \frac{dv_y}{dt} = -\frac{GM_{\odot}y}{r^3}.   
\end{align*}
When extending to three dimensions we simply add two more equations like those above, simply 
replacing $x$ or $y$ by $z$. 

The next thing we want to do is to scale the equations appropriately. When working on an astronomical 
scale we prefer to work with years (yr) as the time unit and AU\footnote{Astronomical units; $1$ AU is
defined as the mean distance between the Earth and the Sun.} as the length unit. If we assume the orbit 
to be circular (which is very close to the truth), the acceleration is given as 
\begin{align*}
a = \frac{v^2}{r} = \frac{F_G}{M_{\text{Earth}}} = \frac{GM_{\odot}}{r^2} \quad \Rightarrow \quad 
v^2 r = GM_{\odot}. 
\end{align*}  
where $r = 1$ AU, while the velocity is 
\begin{align*}
v = \frac{2\pi r}{t} = 2\pi \:\: \text{AU/yr},   
\end{align*}
which means that 
\begin{align*}
GM_{\odot} = 4\pi^2 \:\: \text{AU}^3/\text{yr}^2. 
\end{align*}

\begin{table}
\begin{center}
\begin{tabular}{ccc} \hline\hline
Planet & Mass (kg) & Distance to Sun (AU) \\ \hline 
Earth & $6\times10^{24}$ &  \\
Jupiter & $1.9\times10^{27}$ &  \\
Mars & $6.6\times10^{23}$ &  \\
Venus & $4.9\times10^{24}$ &  \\
Saturn & $5.5\times10^{26}$ & \\ 
Mercury & $3.3\times10^{23}$ & \\ 
Uranus & $8.8\times10^{25}$ & \\ 
Neptune & $1.03\times10^{26}$ &  \\ 
Pluto & $1.31\times10^{22}$ & \\ \hline 
\end{tabular}
\end{center}
\end{table}

\section{Algorithms}

\section{Code}

\section{Results}

\section{Summary and conclusions}

\end{document}

